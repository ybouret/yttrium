\documentclass[aps,12pt]{revtex4}
\usepackage[a4paper]{geometry}
\usepackage{graphicx}
\usepackage{amssymb,amsfonts,amsmath,amsthm}
\usepackage{bm}
\usepackage{pslatex}
\usepackage{chemarr}
\usepackage{mathptmx}
\usepackage{bookman}
\usepackage{esint}
\usepackage{xfrac}  	 
	 
 		 
\begin{document}

\title{Steady-State pH profile}
\maketitle


\begin{equation}
\begin{array}{rclcl}
	\partial_t [H^+]  & = & D_H \Delta [H^+]     &+&\chi_H\\
	\partial_t [HO^-] & = & D_{HO} \Delta [HO^-] &+&\chi_{HO}\\
	\partial_t [AH]   & = & D_{AH} \Delta [AH]   &+&\chi_{AH}\\
	\partial_t [A^-]  & = & D_{A^-} \Delta [A^-] &+& \chi_{A^-}\\
	\partial_t [X^{z_Z}] & = & D_X \Delta [X^{z_X}] &+& 0\\
\end{array}
\end{equation}

\begin{equation}
	\begin{array}{rcll}
	H_2O  & \rightleftharpoons & H^+ + HO^-, & K_w\\
	AH    & \rightleftharpoons & H^+ + A^-,  & K_a\\
	\end{array}
\end{equation}

\begin{equation}
	[H^+]-[HO^-]-[A^-]+Q\equiv0
\end{equation}


\end{document}
