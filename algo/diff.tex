\documentclass[aps,12pt]{revtex4}
\usepackage[a4paper]{geometry}
\usepackage{graphicx}
\usepackage{amssymb,amsfonts,amsmath,amsthm}
\usepackage{bm}
\usepackage{pslatex}
\usepackage{chemarr}
\usepackage{mathptmx}
\usepackage{bookman}

  	 
\begin{document}

\section{intro}

\subsection{Electro Chemical potential}



\begin{equation}
	\mu_i = RT \ln a_i + z_i \mathcal{F} V = RT \ln (\gamma_i[C_i]) + z_i \mathcal{F} V
\end{equation}

Poisson equation:
\begin{equation}
	\Delta V = - \dfrac{\rho}{\epsilon} = -\dfrac{\mathcal{F}}{\epsilon} k_1 \sum_i z_i C_i,\;k_1=1000
\end{equation}
We have $\rho$ in [elementary charges/$m^3$] and $C$ in moles/$dm^3$.

\begin{equation}
	\vec{E} = -\vec{\nabla} V
\end{equation}

\subsection{Particular velocity}
Molecular force:	
\begin{equation}
	\vec{F}_i = - \frac{1}{\mathcal{N}_A} \vec{\nabla} \mu_i
\end{equation}

Molecular motion:
\begin{equation}
	m_i \ddot {\vec{r}} = \vec{F}_i - a_i \vec{v}_i \implies \vec{v}_i \simeq \dfrac{1}{a_i} \vec{F}_i
\end{equation}

\begin{equation}
	\vec{v}_i = -\underbrace{\dfrac{k_bT}{a_i}}_{D_i} \vec{\nabla}(\ln(\gamma_i C_i))  + z_i \underbrace{\dfrac{ q}{a_i}}_{\lambda_i} \vec{E}
\end{equation}

\begin{equation}
	\lambda_i = \frac{q}{k_bT} D_i = \dfrac{\mathcal{F}}{RT} D_i
\end{equation}

\begin{equation}
\boxed{
	\vec{v}_i = D_i \left[ - \vec{\nabla}(\ln(\gamma_i C_i)) + z_i \dfrac{\mathcal{F}}{RT} \vec{E} \right]
}
\end{equation}

\subsection{Continuity equations}

Molar flux:
\begin{equation}
	\vec{J}_i = C_i \vec{v}_i = D_i \left[z_i C_i  \dfrac{\mathcal{F}}{RT} \vec{E} - \left(\vec{\nabla} C_i + C_i \vec{\nabla}\Gamma_i\right) \right],\;\;\Gamma_i = \ln \gamma_i
\end{equation}

Continuity equation:
\begin{equation}
	\partial_t C_i + \mathrm{div} \vec{J}_i = \sigma_i
\end{equation}

Electric continuity:
\begin{equation}
	\partial_t \left(\sum_i z_i C_i\right) + \mathrm{div} \left(\sum_i z_i \vec{J}_i\right) = \sum_i z_i \sigma_i
\end{equation}


\begin{equation}
	\mathrm{div}(f\vec{g}) = f \, \mathrm{div}\vec{g} + \vec{\nabla} f \cdot \vec{g}
\end{equation}

In a constant $D_i$ bulk:
\begin{equation}
\begin{array}{rcl}
\mathrm{div} \vec{J}_i & = & D_i \left[ \dfrac{z_i\mathcal{F}}{RT} 
\left(C_i \mathrm{div}\left(\vec{E}\right) + \vec{\nabla} C_i \cdot \vec{E} \right) 
- \left( \Delta C_i + C_i \Delta \Gamma_i + \vec{\nabla} C_i \cdot \vec{\nabla} \Gamma_i \right)
\right]\\
\\
 & = &  D_i \left[ \dfrac{z_i\mathcal{F}}{RT} 
\left(  \vec{\nabla} C_i \cdot \vec{E} - C_i \Delta V \right) 
- \left( \Delta C_i + C_i \Delta \Gamma_i + \vec{\nabla} C_i \cdot \vec{\nabla} \Gamma_i \right)
\right]\\
\end{array}
\end{equation}

\subsection{1D Three Bodies Problem}

$h,w,y,\Psi$ between $0$ and $L$.
Are known: $h_0,w_0,y_0,\Psi_0$ and $h_l,w_l,y_l,\Psi_l$.

\begin{equation}
	\Delta \Psi = -\kappa_\rho \sum_i z_i C_i
\end{equation}
so that
\begin{equation}
	\dfrac{\Psi_0 - 2\Psi + \Psi_l}{l^2} = -\kappa_\rho \rho
\end{equation}
and
\begin{equation}
	\Psi = \dfrac{1}{2} \left( l^2\kappa_\rho + \Psi_0 + \Psi_l \right)
\end{equation}

For a given concentration:
\begin{itemize}
	\item 
	$C_-  = \frac{1}{2}(C_0+C)$, 
	$\vec{E}_- = -\dfrac{2}{l}\left(\Psi-\Psi_0\right) $,
	$\partial_x C_- = \dfrac{2}{l}(C-C_0)$
	\item 
	$C_+  = \frac{1}{2}(C+C_l)$, 
	$\vec{E}_+ = -\dfrac{2}{l}\left(\Psi_l-\Psi_0\right)$,
	 $\partial_x C_+ = \dfrac{2}{l}(C_l-C)$
\end{itemize}

\begin{itemize}
	\item $J_- = D \left[ z C_- \dfrac{\mathcal{F}E_-}{RT} - \partial_x C_-\right]$
	\item $J_+ = D \left[ z C_+ \dfrac{\mathcal{F}E_+}{RT} - \partial_x C_+\right]$
\end{itemize}


\end{document}

\begin{itemize}
\item $h_0,w_0=K/h_0$, $z_y y_0+h_0-K/h_0=0$
\item $h_l,w_l=K/w_l$, $z_y y_l+h_l-K/w_l=0$
\end{itemize}

We search for the middle position.
\begin{itemize}
	\item $\vec{J}_{i_-} = D_i \left[z_i C_{i_-}  \dfrac{\mathcal{F}}{RT} \vec{E}_- - \vec{\nabla} C_{i_-} \right]$
	\item $\vec{J}_{i_+} = D_i \left[z_i C_{i_+}  \dfrac{\mathcal{F}}{RT} \vec{E}_+ - \vec{\nabla} C_{i_+} \right]$
	\item $C_{i_-}=\dfrac{1}{2}(C_{i_0}+C_i)$, $\vec{\nabla} C_{i_-} = 2(C_i-C_{i_0})/l$
	\item $C_{i_+}=\dfrac{1}{2}(C_{i_l}+C_i)$, $\vec{\nabla} C_{i_+} = 2(C_{i_l}-C_i)/l$
\end{itemize}

$$\partial_t C_i = 0 \implies \vec{J}_{i_-} = \vec{J}_{i_+}$$

$$
	 	\left[z_i C_{i_-}  \dfrac{\mathcal{F}}{RT} \vec{E}_- - \vec{\nabla} C_{i_-} \right] = z_i \dfrac{1}{2} (C_{i_0}+C_i) \zeta_- - 2(C_i-C_{i_0})/l
$$
$$
   \left[z_i C_{i_+}  \dfrac{\mathcal{F}}{RT} \vec{E}_+ - \vec{\nabla} C_{i_+} \right] 
 =  z_i \dfrac{1}{2} (C_{i_l}+C_i) \zeta_+ - 2(C_{i_l}-C_i)/l
$$


\end{document}
