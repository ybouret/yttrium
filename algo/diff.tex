\documentclass[aps,12pt]{revtex4}
\usepackage[a4paper]{geometry}
\usepackage{graphicx}
\usepackage{amssymb,amsfonts,amsmath,amsthm}
\usepackage{bm}
\usepackage{pslatex}
\usepackage{chemarr}
\usepackage{mathptmx}
\usepackage{bookman}

  	 
\begin{document}

\section{intro}

\subsection{Electro Chemical potential}



\begin{equation}
	\mu_i = RT \ln a_i + z_i \mathcal{F} V = RT \ln (\gamma_i[C_i]) + z_i \mathcal{F} V
\end{equation}

Poisson equation:
\begin{equation}
	\Delta V = - \dfrac{\rho}{\epsilon} = -\dfrac{\mathcal{F}}{\epsilon} k_1 \sum_i z_i C_i,\;k_1=1000
\end{equation}
We have $\rho$ in [elementary charges/$m^3$] and $C$ in moles/$dm^3$.

\begin{equation}
	\vec{E} = -\vec{\nabla} V
\end{equation}

\subsection{Particular velocity}
Molecular force:	
\begin{equation}
	\vec{F}_i = - \frac{1}{\mathcal{N}_A} \vec{\nabla} \mu_i
\end{equation}

Molecular motion:
\begin{equation}
	m_i \ddot {\vec{r}} = \vec{F}_i - a_i \vec{v}_i \implies \vec{v}_i \simeq \dfrac{1}{a_i} \vec{F}_i
\end{equation}

\begin{equation}
	\vec{v}_i = -\underbrace{\dfrac{k_bT}{a_i}}_{D_i} \vec{\nabla}(\ln(\gamma_i C_i))  + z_i \underbrace{\dfrac{ q}{a_i}}_{\lambda_i} \vec{E}
\end{equation}

\begin{equation}
	\lambda_i = \frac{q}{k_bT} D_i = \dfrac{\mathcal{F}}{RT} D_i
\end{equation}

\begin{equation}
\boxed{
	\vec{v}_i = D_i \left[ - \vec{\nabla}(\ln(\gamma_i C_i)) + z_i \dfrac{\mathcal{F}}{RT} \vec{E} \right]
}
\end{equation}

\subsection{Continuity equations}

Molar flux:
\begin{equation}
	\vec{J}_i = C_i \vec{v}_i = D_i \left[z_i C_i  \dfrac{\mathcal{F}}{RT} \vec{E} - \left(\vec{\nabla} C_i + C_i \vec{\nabla}\Gamma_i\right) \right],\;\;\Gamma_i = \ln \gamma_i
\end{equation}

Continuity equation:
\begin{equation}
	\partial_t C_i + \mathrm{div} \vec{J}_i = \sigma_i
\end{equation}

Electric continuity:
\begin{equation}
	\partial_t \left(\sum_i z_i C_i\right) + \mathrm{div} \left(\sum_i z_i \vec{J}_i\right) = \sum_i z_i \sigma_i
\end{equation}


\begin{equation}
	\mathrm{div}(f\vec{g}) = f \, \mathrm{div}\vec{g} + \vec{\nabla} f \cdot \vec{g}
\end{equation}

In a constant $D_i$ bulk:
\begin{equation}
\begin{array}{rcl}
\mathrm{div} \vec{J}_i & = & D_i \left[ \dfrac{z_i\mathcal{F}}{RT} 
\left(C_i \mathrm{div}\left(\vec{E}\right) + \vec{\nabla} C_i \cdot \vec{E} \right) 
- \left( \Delta C_i + C_i \Delta \Gamma_i + \vec{\nabla} C_i \cdot \vec{\nabla} \Gamma_i \right)
\right]\\
\\
 & = &  D_i \left[ \dfrac{z_i\mathcal{F}}{RT} 
\left(  \vec{\nabla} C_i \cdot \vec{E} - C_i \Delta V \right) 
- \left( \Delta C_i + C_i \Delta \Gamma_i + \vec{\nabla} C_i \cdot \vec{\nabla} \Gamma_i \right)
\right]\\
\end{array}
\end{equation}

\section{Electroneutrality}

\subsection{Diffusive Strength Field}

\begin{equation}
\begin{array}{|l|r|}
\hline
 \text{} & D[m^2/s]  \\
 \hline
 	H^+	 &9.31\cdot10^{-9}	\\
 	Na^+ &1.33\cdot10^{-9}	 \\
 	K^+	 &1.96\cdot10^{-9}	 \\
 	HO^- &5.27\cdot10^{-9}	 \\
 	Cl^- & 2.03\cdot10^{-9} \\
 	Br^- & 2.01\cdot10^{-9} \\
\hline
\end{array}
\end{equation}
$$
	K_w = 10^{-14}\; M^2
$$

\begin{equation}
	\Omega(\vec r,t) = \sum_i z_i^2 D_i Ci(\vec r, t)
\end{equation}

In water,
\begin{equation}
	\Omega(\vec r,t) \geq [H^+] D_h + D_w K_w/[H^+]  \geq 2 \sqrt{D_h D_w K_w} \approx 1.40 \cdot 10^{-15} \; M.m^2.s^{-1} 
\end{equation}

\subsection{Full Problem}

Let us assume that we start from $\langle \vec Z \vert \vec V \rangle = 0$.
\begin{equation}
	\partial_t \vec C = \vec F ( \vec C )
\end{equation}
Then during $\delta t$ we predict:
\begin{equation}
	\vec C (\vec r, t+\delta t) = \vec C ' (\vec r)
\end{equation}

On a fast scale, the internal charge separation.
We want $\delta \vec C(\vec r)$ such that
\begin{equation}
	\langle \vec Z \vert \vec C' (\vec r) + \delta \vec C(\vec r) \rangle = 0
\end{equation}

\begin{equation}
	\rho(\vec r,\tau) = \langle \vec Z \vert \vec C' (\vec r) + \delta \vec C(\vec r,\tau) \rangle
\end{equation}

\begin{equation}
	\Delta \zeta(\vec r, \tau) = - \kappa_\rho \rho(\vec r,\tau)
\end{equation}

\begin{equation}
	\vec \eta(\vec r, \tau) = - \vec \nabla \zeta(\vec r,\tau)
\end{equation}

\begin{equation}
	\vec \Phi_i(\vec r,\tau) = z_i D_i \left(   C'_i (\vec r) + \delta  C_i(\vec r,\tau) \right) \vec \eta(\vec r, \tau)
\end{equation}

\begin{equation}
	\partial_\tau \delta   C_i(\vec r, \tau) = - \mathrm{div} \; \vec \Phi_i (\vec r,\tau)
\end{equation}


\begin{equation}
	\partial_\tau \delta   C_i(\vec r, \tau) 
	= - z_i D_i 
	\left[
	\left( C'_i (\vec r) + \delta  C_i(\vec r,\tau) \right) \kappa_\rho \rho(\vec r,\tau)
	+ \vec \eta(\vec r,\tau) \cdot \vec \nabla \left( \ldots \right)
	\right]
\end{equation}

\begin{equation}
\begin{array}{rcl}
	\partial_\tau   \rho(\vec r, \tau) &
	= & - \kappa_\rho \left[ \sum_i z_i^2 D_i \left( C'_i (\vec r) + \delta  C_i(\vec r,\tau) \right) \right]   \rho(\vec r,\tau) \\
	 & & - \vec \eta(\vec r,\tau) \cdot \left[\sum_i z_i^2 D_i \vec\nabla \left( C'_i (\vec r) + \delta  C_i(\vec r,\tau) \right) \right]\\
\end{array}
\end{equation}

\subsection{Coupled Approach}

\begin{equation}
	\partial_t \vec C = \vec F ( \vec C ) + \vec \Psi
\end{equation}

\begin{equation}
	\partial_t \rho = \langle \vec Z \vert \vec \partial_t \vec C \rangle = \partial_t \rho_F + \partial_t \rho_\Psi
\end{equation}

\begin{equation}
	\Delta \partial_t \zeta = -\kappa_\rho \partial_t \rho = -\kappa_\rho ( \partial_t \rho_F + \partial_t \rho_\Psi)
\end{equation}

\begin{equation}
	\partial_t \vec \eta = - \vec\nabla \partial_t \zeta = \partial_t \vec\eta_F + \partial_t \vec\eta_\Psi
\end{equation}

\begin{equation}
	\vec \Phi_i = z_i D_i C_i \vec\eta_\Psi
\end{equation}

\begin{equation}
	\Psi_i = - \mathrm{div}\; \vec \Phi_i 
\end{equation}

\begin{equation}
\begin{array}{rcl}
	\partial_t \rho_\Psi & = & \sum_i z_i \Psi_i \\
	& = &  - \sum_i z_i^2 D_i \mathrm{div}\;(C_i \vec \eta_\Psi)\\
	& = &  - \partial_t \rho_F\\
	& = &  - \langle \vec Z \vert \vec F \rangle \\
\end{array}
\end{equation}

\begin{equation}
\begin{array}{rcl}
\partial_t \rho_F & = & - \sum_i z_i^2 D_i \mathrm{div}\;(C_i \vec \eta_F)\\
 & = & - \sum_i z_i^2 D_i \left[ C_i \mathrm{div}\;  \vec \eta_F + \vec \eta_F \cdot \vec \nabla C_i \right]\\
 \end{array}
\end{equation}

\begin{equation}
\partial_t \rho_F  = - \kappa_\rho \left[\sum_i z_i^2 D_i C_i\right] \rho_F - 
\left[\sum_i z_i^2 D_i \vec \nabla C_i \right] \cdot \vec \eta_F
\end{equation}


\subsection{Finite Volumes}

We take a linear part of length $2L$, and a control volume between $L_-=L/2$ and $L_+=3L/2$,
with boundary conditions $\vec C_0$ and $\vec C_L$.

\begin{equation}
	\dfrac{1}{V} \iiint \partial_t C_i(\vec r,t)  \mathrm{d}\, \vec r
	= - \oint  	\vec{J}_i \mathrm{d}\, \vec S + \dfrac{1}{V} \iiint \sigma_i \mathrm{d}\, \vec r
\end{equation}
 
\begin{equation}
	\partial_t C_i = -\dfrac{S}{V}(J_{i+}-J_{i-}) + \sigma_i
\end{equation}

\begin{equation}
	J_{i\pm} = \underbrace{\hat J_{i\pm}}_{-D_i \partial_x C_{i\pm}+\ldots} + z_i D_i C_{i\pm} \eta_\pm
\end{equation}

\begin{equation}
	\partial_t C_i = \sigma_i -\dfrac{1}{L} \left[ \hat J_{i+}- \hat J_{i-} + z_i D_i (C_{i+} \eta_+ - C_{i-} \eta_-) \right] 
\end{equation}

\begin{equation}
	\partial_t C_i = \underbrace{\sigma_i -\dfrac{1}{L} \left[ \hat J_{i+}- \hat J_{i-} \right]}_{F_i} - \dfrac{1}{L}  \left[ z_i D_i (C_{i+} \eta_+ - C_{i-} \eta_-) \right] 
\end{equation}


\begin{equation}
	\sum_i z_i \partial_t C_i = \left(\sum_i z_i \sigma_i\right) 
	-\dfrac{1}{L} \left( \sum_i z_i\left[ \hat J_{i+}- \hat J_{i-} \right]  \right)
	- \dfrac{1}{L}  \left[ \Omega_{i+} \eta_+ - \Omega_{i-} \eta_- \right] 
\end{equation}

\begin{equation}
	\dfrac{1}{L}  \left[ \Omega_{i+} \eta_+ - \Omega_{i-} \eta_- \right]  =
	\left(\sum_i z_i \sigma_i\right) 
	-\dfrac{1}{L} \left( \sum_i z_i\left[ \hat J_{i+}- \hat J_{i-} \right]  \right)
	= \langle \vec Z \vert \vec F \rangle
\end{equation}

 
We look for the smallest field such that 
\begin{equation}
\Omega_{i+} \eta_+ - \Omega_{i-} \eta_- = L \langle \vec Z \vert \vec F \rangle
\end{equation}

\begin{equation}
\begin{array}{rcl}
	\eta_- & = & - L \dfrac{\Omega_{i-}}{ \Omega_{i-}^2 + \Omega_{i+}^2} \langle \vec Z \vert \vec F \rangle \\
	\\
 	\eta_+ & = &  L \dfrac{\Omega_{i+}}{ \Omega_{i-}^2 + \Omega_{i+}^2} \langle \vec Z \vert \vec F \rangle \\
\end{array}
\end{equation}


\begin{equation}
\begin{array}{rcl}
\partial_t C_i & = & F_i 
- \dfrac{1}{L}  \left[ z_i D_i (C_{i+} \eta_+ - C_{i-} \eta_-) \right] \\
& = & F_i - \dfrac{z_i D_i }{ \Omega_{i-}^2 + \Omega_{i+}^2 } \left[ C_{i-} \Omega_{i-} + C_{i+} \Omega_{i+} \right] \langle \vec Z \vert \vec F \rangle
\end{array}
\end{equation}

\end{document}



\subsection{Acting on $\partial_t \vec C$}
We start from $\langle \vec Z \vert \vec C \rangle = 0$ and the physics:
\begin{equation}
	\partial_t \vec C = \vec F \left(\vec C \right)
\end{equation}

The kinetic charge excess will modify each term with $\alpha_i$ so that:
\begin{equation}
		\langle \vec Z \vert \vec \alpha \otimes \vec F \rangle = 0
\end{equation}
We want to minimize the distance of $\vec \alpha $ to the weights $\vec \omega$.

\begin{equation}
	\mathcal L = \dfrac{1}{2} \left( \vec \alpha - \vec\omega\right)^2 - \lambda \langle \vec Z \vert \vec \alpha \otimes \vec F \rangle
\end{equation} 

\begin{equation}
	\partial_{\vec \alpha} \mathcal L = \vec{\alpha} - \vec{\omega} - \lambda \vec Z \otimes \vec F  \implies \vec \alpha = \vec \omega + \lambda \vec Z \otimes \vec F
\end{equation}


\begin{equation}
\langle \vec Z \vert \vec \alpha \otimes \vec F \rangle = 0 =  \langle \vec Z \otimes \vec F \vert \vec \alpha \rangle \implies 
0 = \langle \vec Z \otimes \vec F \vert \vec \omega \rangle + \lambda \left[\vec Z \otimes \vec F\right]^2
\end{equation}

\begin{equation}
	\lambda = - \dfrac{\langle \vec Z \otimes \vec F \vert \vec \omega \rangle}{\left[\vec Z \otimes \vec F\right]^2}
\end{equation}

\begin{equation}
	\vec \alpha = \left( \bm{I} - \dfrac{1}{\left[\vec Z \otimes \vec F\right]^2}\left( \vert \vec Z \otimes \vec F \rangle \langle \vec Z \otimes \vec F \vert \right) 
	\right) \vec\omega
\end{equation}

For any $\vec\omega$, this expression ensure a local charge conservation.

\subsection{Strong condition on $\vec J_i$}

The effect of the reaction field is to compensate exactly for the net electric displacement:

\begin{equation}
	\vec \Phi_i = z_i D_i C_i \vec\eta
\end{equation}
	
\begin{equation}
	\sum_i z_i(\vec J_i + \vec \Phi_i) = \vec 0 = \sum_i z_i \vec J_i + \left(\sum_i z_i^2 D_i C_i\right) \vec\eta
\end{equation}

  
\end{document}
 