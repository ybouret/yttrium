\documentclass[aps,12pt]{revtex4}
\usepackage[a4paper]{geometry}
\usepackage{graphicx}
\usepackage{amssymb,amsfonts,amsmath,amsthm}
\usepackage{bm}
\usepackage{pslatex}
\usepackage{chemarr}
\usepackage{mathptmx}
\usepackage{bookman}

  	 
\begin{document}

\section{intro}

\subsection{Electro Chemical potential}



\begin{equation}
	\mu_i = RT \ln a_i + z_i \mathcal{F} V = RT \ln (\gamma_i[C_i]) + z_i \mathcal{F} V
\end{equation}

Poisson equation:
\begin{equation}
	\Delta V = - \dfrac{\rho}{\epsilon} = -\dfrac{\mathcal{F}}{\epsilon} k_1 \sum_i z_i C_i,\;k_1=1000
\end{equation}
We have $\rho$ in [elementary charges/$m^3$] and $C$ in moles/$dm^3$.

\begin{equation}
	\vec{E} = -\vec{\nabla} V
\end{equation}

\subsection{Particular velocity}
Molecular force:	
\begin{equation}
	\vec{F}_i = - \frac{1}{\mathcal{N}_A} \vec{\nabla} \mu_i
\end{equation}

Molecular motion:
\begin{equation}
	m_i \ddot {\vec{r}} = \vec{F}_i - a_i \vec{v}_i \implies \vec{v}_i \simeq \dfrac{1}{a_i} \vec{F}_i
\end{equation}

\begin{equation}
	\vec{v}_i = -\underbrace{\dfrac{k_bT}{a_i}}_{D_i} \vec{\nabla}(\ln(\gamma_i C_i))  + z_i \underbrace{\dfrac{ q}{a_i}}_{\lambda_i} \vec{E}
\end{equation}

\begin{equation}
	\lambda_i = \frac{q}{k_bT} D_i = \dfrac{\mathcal{F}}{RT} D_i
\end{equation}

\begin{equation}
\boxed{
	\vec{v}_i = D_i \left[ - \vec{\nabla}(\ln(\gamma_i C_i)) + z_i \dfrac{\mathcal{F}}{RT} \vec{E} \right]
}
\end{equation}

\subsection{Continuity equations}

Molar flux:
\begin{equation}
	\vec{J}_i = C_i \vec{v}_i = D_i \left[z_i C_i  \dfrac{\mathcal{F}}{RT} \vec{E} - \left(\vec{\nabla} C_i + C_i \vec{\nabla}\Gamma_i\right) \right],\;\;\Gamma_i = \ln \gamma_i
\end{equation}

Continuity equation:
\begin{equation}
	\partial_t C_i + \mathrm{div} \vec{J}_i = \sigma_i
\end{equation}

Electric continuity:
\begin{equation}
	\partial_t \left(\sum_i z_i C_i\right) + \mathrm{div} \left(\sum_i z_i \vec{J}_i\right) = \sum_i z_i \sigma_i
\end{equation}


\begin{equation}
	\mathrm{div}(f\vec{g}) = f \, \mathrm{div}\vec{g} + \vec{\nabla} f \cdot \vec{g}
\end{equation}

In a constant $D_i$ bulk:
\begin{equation}
\begin{array}{rcl}
\mathrm{div} \vec{J}_i & = & D_i \left[ \dfrac{z_i\mathcal{F}}{RT} 
\left(C_i \mathrm{div}\left(\vec{E}\right) + \vec{\nabla} C_i \cdot \vec{E} \right) 
- \left( \Delta C_i + C_i \Delta \Gamma_i + \vec{\nabla} C_i \cdot \vec{\nabla} \Gamma_i \right)
\right]\\
\\
 & = &  D_i \left[ \dfrac{z_i\mathcal{F}}{RT} 
\left(  \vec{\nabla} C_i \cdot \vec{E} - C_i \Delta V \right) 
- \left( \Delta C_i + C_i \Delta \Gamma_i + \vec{\nabla} C_i \cdot \vec{\nabla} \Gamma_i \right)
\right]\\
\end{array}
\end{equation}

\subsection{Three Bodies Problem}

We start from $\langle \vec Z \vert \vec C \rangle = 0$ and the physics:
\begin{equation}
	\partial_t \vec C = \vec F \left(\vec C \right)
\end{equation}

The kinetic charge excess will modify each term with $\alpha_i$ so that:
\begin{equation}
		\langle \vec Z \vert \vec \alpha \otimes \vec F \rangle = 0
\end{equation}
We want to minimize the distance of $\vec \alpha $ to the weights $\vec \omega$.

\begin{equation}
	\mathcal L = \dfrac{1}{2} \left( \vec \alpha - \vec\omega\right)^2 - \lambda \langle \vec Z \vert \vec \alpha \otimes \vec F \rangle
\end{equation} 

\begin{equation}
	\partial_{\vec \alpha} \mathcal L = \vec{\alpha} - \vec{\omega} - \lambda \vec Z \otimes \vec F  \implies \vec \alpha = \vec \omega + \lambda \vec Z \otimes \vec F
\end{equation}


\begin{equation}
\langle \vec Z \vert \vec \alpha \otimes \vec F \rangle = 0 =  \langle \vec Z \otimes \vec F \vert \vec \alpha \rangle \implies 
0 = \langle \vec Z \otimes \vec F \vert \vec \omega \rangle + \lambda \left[\vec Z \otimes \vec F\right]^2
\end{equation}

\begin{equation}
	\lambda = - \dfrac{\langle \vec Z \otimes \vec F \vert \vec \omega \rangle}{\left[\vec Z \otimes \vec F\right]^2}
\end{equation}

\begin{equation}
	\vec \alpha = \left( \bm{I} - \dfrac{1}{\left[\vec Z \otimes \vec F\right]^2}\left( \vert \vec Z \otimes \vec F \rangle \langle \vec Z \otimes \vec F \vert \right) 
	\right) \vec\omega
\end{equation}

For any $\omega$, this expression ensure a local charge conservation.

  
\end{document}
 