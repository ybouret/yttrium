\documentclass[aps,12pt]{revtex4}
\usepackage[a4paper]{geometry}
\usepackage{graphicx}
\usepackage{amssymb,amsfonts,amsmath,amsthm}
\usepackage{bm}
\usepackage{pslatex}
\usepackage{chemarr}
\usepackage{mathptmx}
\usepackage{bookman}

  	 
\begin{document}

We assume that $\lambda_k$ is close to a given eigenvalue of $\bm{A}$ with a (normalized) eigenvector $\vec{v}_k$:
\begin{equation}
	\bm{A} \vec{v}_k \simeq \lambda_k \vec{v}_k
\end{equation}

\begin{equation}
\begin{array}{rcl}
	\mathcal{L}(\lambda,\vec{v}) & =  &
	\dfrac{1}{2}
	\dfrac{ 
	\langle \bm{A}\vec{v} - \lambda \vec{v} \vert \bm{A}\vec{v} - \lambda \vec{v}\rangle
	}
	{
	\langle \vec{v} \vert \vec{v} \rangle 
	}
	\\
	\\
	 & = & \dfrac{1}{2}\left\lbrack 
	 \dfrac{\langle \vec{v} \vert \bm{A}^T \bm{A} \vert \vec{v} \rangle}{\langle \vec{v} \vert \vec{v} \rangle }
	 -2 \lambda \dfrac{\langle \vec{v} \vert \bm{A} \vert \vec{v} \rangle }{\langle \vec{v} \vert \vec{v} \rangle }
	 +\lambda^2
	 \right\rbrack
\end{array}
\end{equation}

Intermediary:
\begin{equation}
	\partial_{\vec{v}} \left( 
	\dfrac{1}{\langle \vec{v} \vert \vec{v} \rangle } 
	\right) = -\dfrac{2}{ \left(\langle \vec{v} \vert \vec{v} \rangle\right)^2 } \vec{v}
\end{equation}

\begin{equation}
\begin{array}{rcl}
	\partial_{\vec{v}} \left( \underbrace{\dfrac{1}{2} \left[ \langle \vec{v} \vert \bm{A}^T \bm{A} \vert \vec{v} \rangle - 2 \lambda \langle \vec{v} \vert \bm{A} \vert \vec{v} \rangle\right]}_{D} \right) 
	& = &  \bm{A}^T \bm{A}   \vec{v} - \lambda \left( \bm{A}^T + \bm{A} \right) \vec{v} \\
	\\
	& = & \left\lbrack \bm{A}^T \bm{A} - \lambda \left( \bm{A}^T + \bm{A}\right) \right\rbrack \vec{v} \\
\end{array}
\end{equation}

Gradient:
\begin{equation}
	\partial_\lambda \mathcal{L} = \lambda - \dfrac{\langle \vec{v} \vert \bm{A} \vert \vec{v} \rangle }{\langle \vec{v} \vert \vec{v} \rangle }
\end{equation}

\begin{equation}
\begin{array}{rcl}
\partial_{\vec{v}} \mathcal{L} & = & \left( 
	\dfrac{1}{\langle \vec{v} \vert \vec{v} \rangle } 
	\right) \partial_{\vec{v}} D + D \partial_{\vec{v}} \left( 
	\dfrac{1}{\langle \vec{v} \vert \vec{v} \rangle } 
	\right)\\
	\\
	& = & 
	\dfrac{1}{\langle \vec{v} \vert \vec{v} \rangle } 
	 \left\lbrack \bm{A}^T \bm{A} - \lambda \left( \bm{A}^T + \bm{A}\right) \right\rbrack \vec{v}
	  -\dfrac{2D}{ \left(\langle \vec{v} \vert \vec{v} \rangle\right)^2 } \vec{v} \\
	  \\
	  & = & 
	  \dfrac{1}{\langle \vec{v} \vert \vec{v} \rangle } 
	  \left(
	  \left\lbrack \bm{A}^T \bm{A} - \lambda \left( \bm{A}^T + \bm{A}\right) \right\rbrack \vec{v}
	  - 
	  \dfrac{
	  \left[ 
	  \langle \vec{v} \vert \bm{A}^T \bm{A} \vert \vec{v} \rangle - 
	  2 \lambda \langle \vec{v} \vert \bm{A} \vert \vec{v} \rangle
	  \right]
	  }{\langle \vec{v} \vert \vec{v} \rangle}
	  \vec{v}
	  \right)\\
\end{array}
\end{equation}

\end{document}

\begin{equation}
	\partial_\lambda \mathcal{L} = - \langle \vec{v} \vert \bm{A}  - \lambda \bm{I} \vert \vec{v} \rangle
\end{equation}

\begin{equation}
	\partial_{\vec{v}} \mathcal{L} =  (\bm{A}  - \lambda \bm{I})^T  (\bm{A}  - \lambda \bm{I}) \vec{v}
\end{equation}

\begin{equation}
	\lambda_{k+1} = \lambda_k + \alpha \langle \vec{v}_k \vert \bm{A}  - \lambda_k \bm{I} \vert \vec{v}_k \rangle
\end{equation}

\begin{equation}
	\vec{u}_{k+1} = \vec{v}_k - \alpha (\bm{A}  - \lambda_k \bm{I})^T(\bm{A}  - \lambda_k \bm{I}) \vec{v}_k
\end{equation}

\begin{equation}
	\vec{v}_{k+1} = \dfrac{1}{\vert \vec{u}_{k+1} \vert} \vec{u}_{k+1}
\end{equation}

\end{document}