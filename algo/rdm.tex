\documentclass[aps,12pt]{revtex4}
\usepackage[a4paper]{geometry}
\usepackage{graphicx}
\usepackage{amssymb,amsfonts,amsmath,amsthm}
\usepackage{bm}
\usepackage{pslatex}
\usepackage{chemarr}
\usepackage{mathptmx}
\usepackage{bookman}

  	 
\begin{document}

\title{Asymptotic Reaction Diffusion Migration}
\maketitle

\section{Reaction(s)}

Let's assume that we have $N$ equilibria coupling $M$ species.
We define $a_j$ the activity of species $A_j$, and $\nu_{ij}$ the algebraic
coefficient of $A_j$ in the $i-$th equilibria $E_i$:
\begin{equation}
	E_i : \sum_j \nu_{ij} A_j = 0, \;\;K_i = \prod_j a_j^{\nu_{ij}}
\end{equation}
We remove the singularities by defining:
\begin{equation}
	\Xi_i(\vec C) = \underbrace{(K_i \prod_{j,\nu_{ij}<0} a_j^{-\nu_{ij}})}_{reactant(s)} - \underbrace{\prod_{j,\nu_{ij}\geq0} a_j^{\nu_{ij}}}_{product(s)}
\end{equation}
and we define an equilibrium state by:
\begin{equation}
	\vec \Xi ( \vec C ) = \vec 0
\end{equation}

Let us make an external "slow" perturbation 
\begin{equation}
	\vec \sigma \, \mathrm{d} t \in \mathbb R ^M
\end{equation}
Then the system will respond with:
\begin{equation}
	\mathrm{d} \vec C = \bm{\nu}^T \, \mathrm{d}\vec \xi\; (\bm{\nu} \in \mathcal M_{N,M},\;\; \vec \xi \in \mathbb R ^N)
\end{equation}
such that
\begin{equation}
	\vec \Xi(\vec C + \vec \sigma \, \mathrm{d} t + \mathrm{d} \vec C ) = \vec 0 \iff \vec 0 = 
	\bm{\Phi} \times ( \vec \sigma \, \mathrm{d} t + \bm{\nu}^T \, \mathrm{d}\vec \xi)
\end{equation}

\begin{equation}
	\bm{\Phi} = \partial_{\vec C} \vec \Xi \in \mathcal M_{N,M}
\end{equation}
so that, starting from $\vec \Xi ( \vec C ) = \vec 0$, the system evolves locally according to:
\begin{equation}
\boxed{
	\partial_t \vec C = \left[\bm{I}_M - \bm{\nu}^T (\bm \Phi \bm{\nu}^T) ^{-1} \bm \Phi \right] \vec \sigma
	}
\end{equation}
which conserves the charge for valid equilibrium(a).

\section{Diffusion/Migration}

\subsection{Electro Chemical potential}

\begin{equation}
	\mu_i = RT \ln a_i + z_i \mathcal{F} V = RT \ln (\gamma_i[C_i]) + z_i \mathcal{F} V
\end{equation}
where $V$ is the external electric potential and
%Poisson equation:
%\begin{equation}
%	\Delta V = - \dfrac{\rho}{\epsilon} = -\dfrac{\mathcal{F}}{\epsilon} k_1 \sum_i z_i C_i,\;k_1=1000
%\end{equation}
%We have $\rho$ in [elementary charges/$m^3$] and $C$ in moles/$dm^3$.

\begin{equation}
	\vec{E} = -\vec{\nabla} V
\end{equation}
is the external electric field.

\subsection{Particular velocity}
Molecular force:	
\begin{equation}
	\vec{F}_i = - \frac{1}{\mathcal{N}_A} \vec{\nabla} \mu_i
\end{equation}

Molecular motion:
\begin{equation}
	m_i \ddot {\vec{r}} = \vec{F}_i - a_i \vec{v}_i \implies \vec{v}_i \simeq \dfrac{1}{a_i} \vec{F}_i
\end{equation}

\begin{equation}
	\vec{v}_i = -\underbrace{\dfrac{k_bT}{a_i}}_{D_i} \vec{\nabla}(\ln(\gamma_i C_i))  + z_i \underbrace{\dfrac{ q}{a_i}}_{\lambda_i} \vec{E}
\end{equation}

\begin{equation}
	\lambda_i = \frac{q}{k_bT} D_i = \dfrac{\mathcal{F}}{RT} D_i
\end{equation}

\begin{equation}
\boxed{
	\vec{v}_i = D_i \left[ - \vec{\nabla}(\ln(\gamma_i C_i)) + z_i \dfrac{\mathcal{F}}{RT} \vec{E} \right]
}
\end{equation}

\subsection{Continuity equations}

Molar flux:
\begin{equation}
	\vec{J}_i = C_i \vec{v}_i = D_i \left[z_i C_i  \dfrac{\mathcal{F}}{RT} \vec{E} - \left(\vec{\nabla} C_i + C_i \vec{\nabla}\Gamma_i\right) \right],\;\;\Gamma_i = \ln \gamma_i
\end{equation}

Continuity equation:
\begin{equation}
	\partial_t C_i + \mathrm{div} \vec{J}_i = \sigma_i
\end{equation}
which generally doesn't conserve the charge.

We define $\vec \eta$ the local reaction field that modifies the fluxes:
\begin{equation}
	\vec \psi_i = z_i D_i C_i \vec \eta
\end{equation}

Let's find the smallest field in a conservative field.

\subsection{Three Bodies Problem}

\subsubsection{Electric coupling}
We take a linear part of length $2L$, and a control volume between $L_-=L/2$ and $L_+=3L/2$,
with boundary conditions $\vec C_0$ and $\vec C_L$.

\begin{equation}
	\dfrac{1}{V} \iiint \partial_t C_i(\vec r,t)  \mathrm{d}\, \vec r
	= - \oint  	\vec{J}_i \mathrm{d}\, \vec S + \dfrac{1}{V} \iiint \sigma_i \mathrm{d}\, \vec r
\end{equation}
 
\begin{equation}
	\partial_t C_i = -\dfrac{S}{V}(J_{i+}-J_{i-}) + \sigma_i
\end{equation}

\begin{equation}
	J_{i\pm} = \underbrace{\hat J_{i\pm}}_{-D_{i\pm} \partial_x C_{i\pm}+\ldots} + z_i D_{i\pm} C_{i\pm} \eta_\pm
\end{equation}

\begin{equation}
	\partial_t C_i = \sigma_i -\dfrac{1}{L} \left[ \hat J_{i+}- \hat J_{i-} + z_i (D_{i+} C_{i+} \eta_+ - D_{i-}C_{i-} \eta_-) \right] 
\end{equation}
that we rewrite as:
\begin{equation}
	\partial_t C_i =
	 \underbrace{\sigma_i -\dfrac{1}{L} \left[ \hat J_{i+}- \hat J_{i-} \right]}_{F_i} 
	 - \dfrac{1}{L}  \left[ z_i (D_{i+} C_{i+} \eta_+ - D_{i-}C_{i-} \eta_-)\right] 
\end{equation}


\begin{equation}
	\sum_i z_i \partial_t C_i =  \langle \vec Z \vert \vec F \rangle - \dfrac{1}{L} \left[ \Omega_+ \eta_{+} - \Omega_- \eta_-\right]
\end{equation}

\begin{equation}
	\Omega_\pm = \sum_i z_i^2 D_{i\pm} C_{i\pm}
\end{equation}
The smallest field is 
\begin{equation}
	\eta_\pm = \pm \dfrac{L \langle \vec Z \vert \vec F \rangle \Omega\pm}{\Omega_-^2+\Omega_+^2}
\end{equation}

\begin{equation}
	\partial_t C_i = F_i - \underbrace{\dfrac{z_i}{\Omega_{i+}^2 + \Omega_{i-}^2} \left[ D_{i+} C_{i+} \Omega_+ + D_{i-} C_{i-} \Omega_- \right]}_{\omega_i} \langle \vec Z \vert \vec F \rangle
\end{equation}

We check that $\langle \vec Z \vert \vec \omega \rangle \equiv 1$.
\begin{equation}
\boxed{
	\partial_t \vec C = \left[\bm I_M - \vert\vec \omega \rangle \langle \vec Z \vert \right] \vec F
	}
\end{equation}
We now expand with the chemical part:


\subsubsection{Electrochemical Coupling}

\begin{equation}
	\partial_t \vec C = \left[\bm{I}_M - \bm{\nu}^T (\bm \Phi \bm{\nu}^T) ^{-1} \bm \Phi \right] \left[\bm I_M - \vert\vec \omega \rangle \langle \vec Z \vert \right] \vec F
\end{equation}


\subsubsection{Three Bodies Saline Solution}

We have water with $NaCl$, four components, one reaction.
We define the left and right side by their osmolarity and pH.

\begin{equation}
\left[\bm{I}_M - \bm{\nu}^T (\bm \Phi \bm{\nu}^T) ^{-1} \bm \Phi \right]
=
\begin{bmatrix}
\frac{h}{w+h} & - \frac{h}{w+h} & 0 & 0\\
-\frac{w}{w+h} & \frac{w}{w+h}  & 0 & 0\\
0 & 0 & 1 & 0\\
0 & 0 & 0 & 1\\
\end{bmatrix}
\end{equation}

\begin{equation}
\left[\bm I_M - \vert\vec \omega \rangle \langle \vec Z \vert \right] =
\begin{bmatrix}
1-\omega_h & \omega_h & -\omega_h & \omega_h\\
-\omega_w  & 1+\omega_w & -\omega_w & \omega_w\\
-\omega_{Na}  & \omega_{Na} & 1-\omega_{Na} & \omega_{Na}\\
-\omega_{Cl}  & \omega_{Cl} & -\omega_{Cl} & 1+\omega_{Cl}\\
\end{bmatrix}
\end{equation}

\section{Data}

\begin{equation}
\begin{array}{|l|r|}
\hline
 \text{} & D[m^2/s]  \\
 \hline
 	H^+	 & 9.31\cdot10^{-9}	\\
 	Na^+ & 1.33\cdot10^{-9}	 \\
 	K^+	 & 1.96\cdot10^{-9}	 \\
 	HO^- & 5.27\cdot10^{-9}	 \\
 	Cl^- & 2.03\cdot10^{-9} \\
 	Br^- & 2.01\cdot10^{-9} \\
\hline
\end{array}
\end{equation}
$$
	K_w = 10^{-14}\; M^2
$$


\end{document}
