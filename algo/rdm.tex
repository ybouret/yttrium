\documentclass[aps,12pt]{revtex4}
\usepackage[a4paper]{geometry}
\usepackage{graphicx}
\usepackage{amssymb,amsfonts,amsmath,amsthm}
\usepackage{bm}
\usepackage{pslatex}
\usepackage{chemarr}
\usepackage{mathptmx}
\usepackage{bookman}
\usepackage{esint}
\usepackage{xfrac}  	 
	 
\newcommand{\half}{\frac{1}{2}}
	 
\begin{document}

\title{Asymptotic Reaction Diffusion Migration}
\maketitle

\section{Reaction(s)}

\subsection{Notations}
Let's assume that we have $N$ equilibria coupling $M$ species.
We define $a_j$ the activity of species $A_j$, and $\nu_{ij}$ the algebraic
coefficient of $A_j$ in the $i-$th equilibria $E_i$:
\begin{equation}
	E_i : \sum_j \nu_{ij} A_j = 0, \;\;K_i = \prod_j a_j^{\nu_{ij}}
\end{equation}
We remove the singularities by defining:
\begin{equation}
	\Xi_i(\vec C) = \underbrace{(K_i \prod_{j,\nu_{ij}<0} a_j^{-\nu_{ij}})}_{reactant(s)} - \underbrace{\prod_{j,\nu_{ij}\geq0} a_j^{\nu_{ij}}}_{product(s)}
\end{equation}
and we define an equilibrium state by:
\begin{equation}
	\vec \Xi ( \vec C ) = \vec 0
\end{equation}

\subsection{Perturbation}
Let us make an external "slow" perturbation 
\begin{equation}
	\vec \sigma \, \mathrm{d} t \in \mathbb R ^M
\end{equation}
Then the system will respond with:
\begin{equation}
	\mathrm{d} \vec C = \bm{\nu}^T \, \mathrm{d}\vec \xi\; (\bm{\nu} \in \mathcal M_{N,M},\;\; \vec \xi \in \mathbb R ^N)
\end{equation}
such that
\begin{equation}
	\vec \Xi(\vec C + \vec \sigma \, \mathrm{d} t + \mathrm{d} \vec C ) = \vec 0 \iff \vec 0 = 
	\bm{\Phi} \times ( \vec \sigma \, \mathrm{d} t + \bm{\nu}^T \, \mathrm{d}\vec \xi)
\end{equation}

\begin{equation}
	\bm{\Phi} = \partial_{\vec C} \vec \Xi \in \mathcal M_{N,M}
\end{equation}
so that, starting from $\vec \Xi ( \vec C ) = \vec 0$, the system evolves locally according to:
\begin{equation}
\boxed{
	\partial_t \vec C = \left[\bm{I}_M - \bm{\nu}^T (\bm \Phi \bm{\nu}^T) ^{-1} \bm \Phi \right] \vec \sigma
	}
\end{equation}
which conserves the charge for valid equilibrium(a).

 
 	
\subsection{Solving}
\subsubsection{1D}
We know that:
\begin{equation}
	\Xi_i(\vec C_0 + \xi_i^\star \vec \nu _i) = \Xi_i(\vec C_0 + \delta \vec C_i^\star) = \Xi_i(\vec C_i^\star) = 0
\end{equation}
and the expansion is:
\begin{equation}
\begin{array}{rl}
	& \Xi_i(\vec C_0 + \sum_j \xi_j \vec \nu_j)\\
=  &\Xi_i(\vec C_0 +  \xi_i^\star \vec \nu _i +  (\xi_i-\xi_i^\star) \vec \nu_i + \sum_{j\not=i} \xi_j \vec \nu_j)\\
\simeq & \langle \vec\Phi_i^\star \vert \vec\nu_i\rangle (\xi_i-\xi_i^\star) +  \langle \vec\Phi_i^\star \vert \sum_{j\not=i} \xi_j \vec \nu_j \rangle \\
\end{array}
\end{equation}
  	
By definition:
\begin{equation}
	\forall i, \;\; \langle \vec\Phi_i^\star \vert \vec\nu_i\rangle < 0
\end{equation}	
	
\subsubsection{Virtual Rates}	

Each equilibrium produces a molar rate proportional to $\Xi_i$. More precisely, the rate is proportional to a time constant
and the concentration:
\begin{equation}
	\dfrac{\Xi_i(\vec C)}{ - \langle \vec\Phi_i^\star \vert \vec\nu_i\rangle }
\end{equation}
where $\vec\Phi_i^\star$ depends on $\vec C$.\\
As we look for the stationary point this is equivalent to assume that:
\begin{equation}
	\dfrac{\Xi_i(\vec C)}{ - \langle \vec\Phi_i^\star \vert \vec\nu_i\rangle } \simeq \xi_i^\star(\vec C)
\end{equation}
This expression is an intrinsic scaling in concentration, so that we assume that the time constant is  the same for
all the equilibria, and we use the reduced (unit) time over ALL the equilibria:

\begin{equation}
\partial_\tau \vec C = \bm{\nu}^T \vec \xi^\star(\vec C) = \vec F(\vec C)
\end{equation}

\subsubsection{Evolving}
Stiff problem
\begin{equation}
	\vec C_{n+1}  \simeq  \vec C_{n} + \tau \vec F(\vec C_{n+1})
\end{equation}
So the next set of concentration is given by the solution of:
\begin{equation}
	\vec Y = \vec C_{n} + \tau \vec F(\vec Y)
\end{equation}
\begin{equation}
	\vec Y_k + \delta \vec Y = \vec C_{n} + \tau \vec F(\vec Y_k + \delta \vec Y) = \vec C_{n} + \tau \vec F(\vec C_n + \vec Y_k + \delta \vec Y - \vec C_n)
\end{equation}

\begin{equation}
	\left[\vec Y_k + \delta \vec Y - \vec C_{n}\right] \simeq \tau \left( \vec F(\vec C_n) + \partial_{\vec C} \vec{F}_n \left[\vec Y_k + \delta \vec Y - \vec C_{n}\right] \right)
\end{equation}

 
\subsubsection{Jacobian}

\begin{equation}
	\Xi_i(\vec C_0 +  \delta \vec C + (\delta\xi_i^\star +\xi_i^\star) \vec \nu _i) = 0
	\iff
	\langle \vec \Phi_i^\star \vert \delta \vec C\rangle + \langle \vec \Phi_i^\star \vert \vec\nu_i \rangle \delta \xi_i^\star = 0
\end{equation}

\begin{equation}
	\delta \xi_i^\star = - \dfrac{\langle \vec \Phi_i^\star \vert \delta \vec C\rangle}{\langle \vec \Phi_i^\star \vert \vec\nu_i \rangle}
\end{equation}

 \begin{equation}
 \bm{\Delta}^\star = \mathrm{diag}\left(\ldots,\dfrac{-1}{\langle \vec \Phi_i^\star \vert \vec\nu_i \rangle},\ldots\right)
\end{equation}

\begin{equation}
	\partial_{\vec C} \vec \xi^\star = \bm{\Delta}^\star \bm{\Phi}^\star 
\end{equation}

\begin{equation}
	\partial_{\vec C} \vec F = \bm{\nu}^T \bm{\Delta}^\star \bm{\Phi}^\star 
\end{equation}

\subsubsection{Diagonally Dominant Matrix}

Let's say we have a matrix $\bm{J}$. On which condition on $\tau\geq0$ the matrix $\bm{I}-\tau\bm{J}$ is diagonally dominant ?

\begin{equation}
 \iff \forall i, \;\; \vert 1 - \tau J_{ii} \vert > \tau \left( \sum_{j\not=i} \vert J_{ij} \vert \right)
\end{equation}

\begin{itemize}
\item $J_{ii}\leq 0$ so that
\begin{equation}
	\vert 1 - \tau J_{ii} \vert = 1+(-J_{ii}) \tau = 1 + \vert J_{ii} \vert \tau
\end{equation}
\begin{equation}
	\vert 1 - \tau J_{ii} \vert > \tau \left( \sum_{j\not=i} \vert J_{ij} \vert \right) 
	\iff  1 > \tau \left[ \left( \sum_{j\not=i} \vert J_{ij} \vert \right) - \vert J_{ii} \vert \right]
\end{equation}
\end{itemize}



\subsection{Conservation}

A conservation law is a vector of {\bf positive} coefficients $\vec \alpha$ such that:
\begin{equation}
	\langle \vec \alpha \vert \vec C \rangle \geq 0
\end{equation}
is conserved during shifts in equilibria.\\
Let's assume that
\begin{equation}
	\langle \vec \alpha \vert \vec C \rangle = -X_s < 0 
\end{equation}
Then we will inject the smallest $\delta \vec C$ such that:
\begin{equation}
	\langle \vec \alpha \vert \delta \vec C \rangle = X_s
\end{equation}

\begin{equation}
	\delta \vec C = \dfrac{X_s}{\vec\alpha^2} \vec{\alpha}
\end{equation}

To preserve numeric symmetry
\begin{equation}
	\vec C_{new} = \vec C + \delta \vec C = \dfrac{1}{\vec \alpha^2}\left[\vec \alpha^2 \bm{I} - \vert \vec \alpha \rangle \langle \vec \alpha \vert \right] \vec C
\end{equation}

 	

\section{Diffusion/Migration}

\subsection{Electro Chemical potential}

\begin{equation}
	\mu_i = RT \ln a_i + z_i \mathcal{F} V = RT \ln (\gamma_i[C_i]) + z_i \mathcal{F} V
\end{equation}
where $V$ is the external electric potential and
%Poisson equation:
%\begin{equation}
%	\Delta V = - \dfrac{\rho}{\epsilon} = -\dfrac{\mathcal{F}}{\epsilon} k_1 \sum_i z_i C_i,\;k_1=1000
%\end{equation}
%We have $\rho$ in [elementary charges/$m^3$] and $C$ in moles/$dm^3$.

\begin{equation}
	\vec{E} = -\vec{\nabla} V
\end{equation}
is the external electric field.

\subsection{Particular velocity}
Molecular force:	
\begin{equation}
	\vec{F}_i = - \frac{1}{\mathcal{N}_A} \vec{\nabla} \mu_i
\end{equation}

Molecular motion:
\begin{equation}
	m_i \ddot {\vec{r}} = \vec{F}_i - a_i \vec{v}_i \implies \vec{v}_i \simeq \dfrac{1}{a_i} \vec{F}_i
\end{equation}

\begin{equation}
	\vec{v}_i = -\underbrace{\dfrac{k_bT}{a_i}}_{D_i} \vec{\nabla}(\ln(\gamma_i C_i))  + z_i \underbrace{\dfrac{ q}{a_i}}_{\lambda_i} \vec{E}
\end{equation}

\begin{equation}
	\lambda_i = \frac{q}{k_bT} D_i = \dfrac{\mathcal{F}}{RT} D_i
\end{equation}

\begin{equation}
\boxed{
	\vec{v}_i = D_i \left[ - \vec{\nabla}(\ln(\gamma_i C_i)) + z_i \dfrac{\mathcal{F}}{RT} \vec{E} \right]
}
\end{equation}

\subsection{Continuity equations}

Molar flux:
\begin{equation}
	\vec{J}_i = C_i \vec{v}_i = D_i \left[z_i C_i  \dfrac{\mathcal{F}}{RT} \vec{E} - \left(\vec{\nabla} C_i + C_i \vec{\nabla}\Gamma_i\right) \right],\;\;\Gamma_i = \ln \gamma_i
\end{equation}

Continuity equation:
\begin{equation}
	\partial_t C_i + \mathrm{div} \vec{J}_i = \sigma_i
\end{equation}
which generally doesn't conserve the charge.

We define $\vec \eta$ the local reaction field that modifies the fluxes:
\begin{equation}
	\vec \psi_i = z_i D_i C_i \vec \eta
\end{equation}

Let's find the smallest field in a conservative field.

\subsection{Three Bodies Problem}

\subsubsection{Electric coupling}
We take a linear part of length $2L$, and a control volume between $L_-=L/2$ and $L_+=3L/2$,
with boundary conditions $\vec C_0$ and $\vec C_L$.

\begin{equation}
	\dfrac{1}{V} \iiint \partial_t C_i(\vec r,t)  \mathrm{d}\, \vec r
	= - \oiint  	\vec{J}_i \mathrm{d}\, \vec S + \dfrac{1}{V} \iiint \sigma_i \mathrm{d}\, \vec r
\end{equation}
 
\begin{equation}
	\partial_t C_i = -\dfrac{S}{V}(J_{i+}-J_{i-}) + \sigma_i
\end{equation}

\begin{equation}
	J_{i\pm} = \underbrace{\hat J_{i\pm}}_{-D_{i\pm} \partial_x C_{i\pm}+\ldots} + z_i D_{i\pm} C_{i\pm} \eta_\pm
\end{equation}

\begin{equation}
	\partial_t C_i = \sigma_i -\dfrac{1}{L} \left[ \hat J_{i+}- \hat J_{i-} + z_i (D_{i+} C_{i+} \eta_+ - D_{i-}C_{i-} \eta_-) \right] 
\end{equation}
that we rewrite as:
\begin{equation}
	\partial_t C_i =
	 \underbrace{\sigma_i -\dfrac{1}{L} \left[ \hat J_{i+}- \hat J_{i-} \right]}_{F_i} 
	 - \dfrac{1}{L}  \left[ z_i (D_{i+} C_{i+} \eta_+ - D_{i-}C_{i-} \eta_-)\right] 
\end{equation}


\begin{equation}
	\sum_i z_i \partial_t C_i =  \langle \vec Z \vert \vec F \rangle - \dfrac{1}{L} \left[ \Omega_+ \eta_{+} - \Omega_- \eta_-\right]
\end{equation}

\begin{equation}
	\Omega_\pm = \sum_i z_i^2 D_{i\pm} C_{i\pm}
\end{equation}
The smallest field is 
\begin{equation}
	\eta_\pm = \pm \dfrac{L \langle \vec Z \vert \vec F \rangle \Omega\pm}{\Omega_-^2+\Omega_+^2}
\end{equation}

\begin{equation}
	\partial_t C_i = F_i - \underbrace{\dfrac{z_i}{\Omega_{i+}^2 + \Omega_{i-}^2} \left[ D_{i+} C_{i+} \Omega_+ + D_{i-} C_{i-} \Omega_- \right]}_{\omega_i} \langle \vec Z \vert \vec F \rangle
\end{equation}

We check that $\langle \vec Z \vert \vec \omega \rangle \equiv 1$.
\begin{equation}
\boxed{
	\partial_t \vec C = \left[\bm I_M - \vert\vec \omega \rangle \langle \vec Z \vert \right] \vec F
	}
\end{equation}
We now expand with the chemical part:


\subsubsection{Electrochemical Coupling}

\begin{equation}
	\partial_t \vec C = \left[\bm{I}_M - \bm{\nu}^T (\bm \Phi \bm{\nu}^T) ^{-1} \bm \Phi \right] \left[\bm I_M - \vert\vec \omega \rangle \langle \vec Z \vert \right] \vec F
\end{equation}


\subsubsection{Three Bodies Saline Solution}

We have water with $NaCl$, four components, one reaction.
We define the left and right side by their osmolarity and pH.

\begin{equation}
\left[\bm{I}_M - \bm{\nu}^T (\bm \Phi \bm{\nu}^T) ^{-1} \bm \Phi \right]
=
\begin{bmatrix}
\frac{h}{w+h} & - \frac{h}{w+h} & 0 & 0\\
-\frac{w}{w+h} & \frac{w}{w+h}  & 0 & 0\\
0 & 0 & 1 & 0\\
0 & 0 & 0 & 1\\
\end{bmatrix}
\end{equation}

\begin{equation}
\left[\bm I_M - \vert\vec \omega \rangle \langle \vec Z \vert \right] =
\begin{bmatrix}
1-\omega_h & \omega_h & -\omega_h & \omega_h\\
-\omega_w  & 1+\omega_w & -\omega_w & \omega_w\\
-\omega_{Na}  & \omega_{Na} & 1-\omega_{Na} & \omega_{Na}\\
-\omega_{Cl}  & \omega_{Cl} & -\omega_{Cl} & 1+\omega_{Cl}\\
\end{bmatrix}
\end{equation}

\section{Data}

\begin{equation}
\begin{array}{|l|r|}
\hline
 \text{} & D[m^2/s]  \\
 \hline
 	H^+	 & 9.31\cdot10^{-9}	\\
 	Na^+ & 1.33\cdot10^{-9}	 \\
 	K^+	 & 1.96\cdot10^{-9}	 \\
 	HO^- & 5.27\cdot10^{-9}	 \\
 	Cl^- & 2.03\cdot10^{-9} \\
 	Br^- & 2.01\cdot10^{-9} \\
\hline
\end{array}
\end{equation}
$$
	K_w = 10^{-14}\; M^2
$$

\section{1D}

We have 
\begin{equation}
	\dfrac{1}{V_k} \partial_t \iiint_{V_k} [A_i] \mathrm d\vec r = \dfrac{1}{V_k} \iiint_{V_k} \left[ \mathrm{div} \vec J_i + \mathrm{div} \vec \Psi _i + \sigma_i \right] \mathrm d \vec r
\end{equation}

\begin{equation}
	\partial_t [A_i]_k = \dfrac{1}{V_k}\oiint_{S_k}  \left[ \vec J_i + \vec \Psi_i \right] \cdot \mathrm d \vec S + \sigma_{i,k}
\end{equation}


\begin{equation}
	\partial_t  [A_i]_k = \dfrac{1}{L_k} \left[ \hat J_{i,k+\half} - \hat J_{i,k-\half} \right] 
	+ \dfrac{z_i}{L_k}\left[ D_{i,k+\half} C_{i,k+\half} \eta_{k+\half} -  D_{i,k-\half} C_{i,k-\half} \eta_{k-\half}\right]+ \sigma_{i,k}
\end{equation}

\begin{equation}
	F_{i,k} =  \dfrac{1}{L_k} \left[ \hat J_{i,k+\half} - \hat J_{i,k-\half} \right]  + \sigma_{i,k}
\end{equation}

\begin{equation}
	\Omega_{\square} = \sum_i z_i^2 D_{i,\square} C_{i,\square}
\end{equation}

\begin{equation}
	\sum_i z_i \partial_t [A_i]_k = 0 \iff \langle \vec Z \vert \vec F_k \rangle
	 + \dfrac{1}{L_k}\left[\Omega_{k+\half} \eta_{k+\half} - \Omega_{k-\half} \eta_{k-\half}\right]
\end{equation}

Let's say we have $n$ vertices labeled with $k\in[1:n]$, then we have $n+1$ interfaces labeled as $\dfrac{2k+1}{2},\;k\in[0:n]$.
We define
\begin{equation}
	\vec \eta \in \mathbb R ^ {n+1}
\end{equation}

\begin{equation}
	\mathcal L = \dfrac{1}{2} \vec \eta ^2 - \sum_k \lambda_k \left(\langle \vec Z \vert \vec F_k \rangle
	 + \dfrac{1}{L_k}\left[\Omega_{k+\half} \eta_{k+\half} - \Omega_{k-\half} \eta_{k-\half}\right]\right)
\end{equation}

\begin{equation}
	\mathcal L = \dfrac{1}{2} \vec \eta ^2 - \langle \vec \lambda \vert \vec{ZF} - 
	\left[\mathfrak D (\vec L)^{-1} \bm M_\nabla \mathfrak D(\vec\Omega)\right] \vec \eta \rangle
\end{equation}

\begin{equation}
	\partial_{\vec \eta} L = \vec \eta +\left[ D(\vec\Omega) \bm M_\nabla ^T D (\vec L)^{-1} \right] \vec \lambda
	\implies \vec \eta = - \left[ \mathfrak D(\vec\Omega) \bm M_\nabla ^T \mathfrak D (\vec L)^{-1} \right] \vec \lambda
\end{equation}

\begin{equation}
	0 = \vec {ZF} + \left[\mathfrak D (\vec L)^{-1} \bm M_\nabla \mathfrak D(\vec\Omega)\right] 
	\left[ \mathfrak D(\vec\Omega) \bm M_\nabla ^T \mathfrak D (\vec L)^{-1} \right] \vec \lambda
\end{equation}

\begin{equation}
	\vec \eta = - \left[ \mathfrak D(\vec\Omega) \bm M_\nabla ^T \mathfrak D (\vec L)^{-1} \right] \left( \left[\mathfrak D (\vec L)^{-1} \bm M_\nabla \mathfrak D(\vec\Omega)\right] \left[ \mathfrak D(\vec\Omega) \bm M_\nabla ^T \mathfrak D (\vec L)^{-1} \right]\right) ^{-1} \vec{ZF}
\end{equation}

\end{document}
