\documentclass[aps,11pt]{revtex4}
\usepackage{graphicx}
\usepackage{amssymb,amsfonts,amsmath,amsthm}
\usepackage{chemarr}
\usepackage{bm}
\usepackage{pslatex}
\usepackage{mathptmx}
\usepackage{xfrac}
\usepackage{bookman}

\newcommand{\mychem}[1]{\mathtt{#1}}
\newcommand{\myconc}[1]{\left\lbrack{#1}\right\rbrack}
\newcommand{\plus}{\mychem{+}}
\newcommand{\proton}{\mychem{H}^\plus}

\begin{document}

\title{Fluorimetry}
\maketitle

\section{Local Fluorescence Intensity from Chemistry}
\noindent Let us assume that we have a fluorescent probe with an acidic form $\alpha$ and a basic form $\beta$ related by the following equilibrium:
\begin{equation}
	\alpha \xrightleftharpoons{} \beta + \proton, \;\; K = \dfrac{\myconc{\beta}\myconc{\proton}}{ \myconc{\alpha} }.
\end{equation}
If the local concentration of the probe is $C_0$, then the concentrations of both specie are, as a function of $h   =   \myconc{\proton} $:
\begin{equation}
\left\lbrace
\begin{array}{rcl}
	
	 \myconc{\alpha}  & = & C_0 \dfrac{h}{K+h}\\
	 \\
	 \myconc{\beta}   & = & C_0 \dfrac{K}{K+h}.\\
\end{array}
\right.
\end{equation}
We now   define the respective fluorescence intensities of each species for a wavelength $\lambda$ as $\phi_\alpha(\lambda)$ and $\phi_\beta(\lambda)$,
so that the total fluorescence intensity emitted by a volume $V$ is:
\begin{equation}
\begin{array}{rcl}
	I(h,\lambda,C_0) & = & V \left( \phi_\alpha(\lambda) \myconc{\alpha}  + \phi_\beta(\lambda) \myconc{\beta} \right)\\
	\\
	& = & V C_0 \left[ \phi_\alpha(\lambda) \dfrac{h}{K+h} + \phi_\beta(\lambda) \dfrac{K}{K+h}\right].\\
\end{array}
\end{equation}
In particular, we define of the isobestic wavelength $\lambda_{iso}$ such that:
\begin{equation}
\phi_\alpha(\lambda_{iso})=\phi_\beta(\lambda_{iso})=\phi_{iso}.
\end{equation}
For this particular wavelength, we get:
\begin{equation}
I(h,\lambda_{iso},C_0) = VC_0\phi_{iso},
\end{equation}
which is independent of $h$ and is used as a normalisation factor to define the measured fluorescence, for a chosen wavelength $\lambda_0$:
\begin{equation}
	F(h) = \dfrac{I(h,\lambda_0,C_0)}{I(h,\lambda_{iso},C_0)} = \dfrac{h\Phi_\alpha + K\Phi_\beta}{h+K},
\end{equation}
with:
\begin{equation}
	\Phi_\alpha = \dfrac{\phi_\alpha(\lambda_0)}{\phi_{iso}},\;\;\Phi_\beta = \dfrac{\phi_\beta(\lambda_0)}{\phi_{iso}}.
\end{equation}

\section{Calibration}
The measure fluorescence depends on two previous coefficients \textit{(fluorescence factors)} that are computed using two different, known $h$ values, for example $h_a$ (for acidic conditions) and $h_b$ (for basic conditions).
In that case:
\begin{equation}
\left\lbrace
	\begin{array}{rcl}
	F(h_a) = F_a & = & \dfrac{h_a\Phi_\alpha + K\Phi_\beta}{h_a+K}\\
	\\
	F(h_b) = F_b & = & \dfrac{h_b\Phi_\alpha + K\Phi_\beta}{h_b+K}.\\
	\end{array}
\right.
\end{equation}
We obtain the fluorescence factors:
\begin{equation}
\left\lbrace
\begin{array}{rcl}
	\Phi_\alpha & = & \dfrac{1}{h_a-h_b}\left[ K(F_a-F_b) + (F_ah_a-F_bh_b)\right]\\
	\\
	\Phi_\beta & = & \dfrac{1}{h_a-h_b}\left[ \dfrac{h_ah_b}{K} (F_a-F_b) + K(F_bh_a-F_ah_b)\right].\\
\end{array}
\right.
\end{equation}

\section{pH Expression}
We deduce the proton concentration as:
\begin{equation}
h=K\left[\dfrac{\Phi_\beta-F}{F-\Phi_\alpha}\right].
\end{equation}
Accordingly, we express the local pH as a function of the measured fluorescence:
\begin{equation}
	\mathrm{pH} = \mathrm{pK} + \log  \left[ \dfrac{F-\Phi_\alpha}{\Phi_\beta-F}\right].
\end{equation}
Using this expression, the pH is evaluated for any fluorescence between $F_a$ and $F_b$ without any singularity.
Obviously, one must wisely choose
the calibration range to avoid the saturations effects: namely, the fluorescence should still vary with $h$ around $h_a$ and $h_b$, while keeping $h_a$ and $h_b$ as separated as possible.
\end{document}
